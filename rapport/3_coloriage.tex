\begin{algorithm}[caption={3-coloriage}] \label{alg:3color}
    Fonction 3-coloriage(G : graphe, S : sommet,
        couleur_dispo_depart : $ensemble\ de\ couleur$) : <$sommet$, $ensemble\ de\ couleur$>
    Variables
        couleur_voisins : $ensemble\ de\ couleur$
        couleur_dispo_fin : $ensemble\ de\ couleur$
        couleur_dispo_successeurs : $ensemble\ de\ couleur$
        res : <$sommet$, $ensemble\ de\ couleur$>
    Debut
        res $\gets$ <null, null>
        couleur_dispo_fin $\gets$ copie(couleur_dispo_depart)
        couleur_voisins $\gets$ ensemble_vide()
        Si (S = null) Alors
            S $\gets$ plusPetitSommet(G)
        FinSi
        Si (S.couleur $\ne$ null) Alors
            Pour chaque sommet V dans voisins(G, S) Faire
                Si (non contient(couleur_voisins, V.couleur)) Alors
                    ajout(couleur_voisins, V.couleur)
                FinSi
            FinPour
            Pour chaque couleur C dans couleur_dispo_depart Faire
                Si (contient(couleur_voisins, C)) Alors
                    retire(couleur_dispo_fin, C)
                FinSi
            FinPour
            Si (non est_vide(couleur_dispo)) Alors
                S.couleur $\gets$ couleur_dispo_depart[0]
                couleur_dispo_successeurs $\gets$ ensemble_vide()
                ajout(couleur_dispo_successeurs, Rouge)
                ajout(couleur_dispo_successeurs, Vert)
                ajout(couleur_dispo_successeurs, Bleu)
                retire(couleur_dispo_successeurs, S.couleur)
                Pour chaque sommet V dans voisins(G, S) Faire
                    res $\gets$ 3-coloriage(G, V, couleur_dispo_successeur)
                    Si (res $\ne$ null) Alors
                        retourner res
                FinPour
            Sinon
                res $\gets$ <S, couleur_voisins>
            FinSi
        FinSi
        retourner res
    Fin
\end{algorithm}