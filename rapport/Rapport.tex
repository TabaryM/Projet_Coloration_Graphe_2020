%! Author = Newtton
%! Date = 04/01/2021

% Preamble
\documentclass[a4paper, 11pt]{article}
\usepackage[french]{babel}
\title{Algorithmique et complexité Projet sur les arbres couvrants}
\author{Paul-Emile Watelot \and Mathieu Tabary}
\date{4 janvier 2021}

% Packages
% Symboles
\usepackage{amsmath}

% Graphes
\usepackage{pgf}
\usepackage{tikz}
\usetikzlibrary{positioning}

% Mise en page
\usepackage{multicol}

\usepackage{color}
\usepackage{listings}
\usepackage{caption}

%Description de pseudocode
%Code trouvé ici : https://tex.stackexchange.com/questions/111116/what-is-the-best-looking-pseudo-code-package/218450#218450
\newcounter{nalg} % defines algorithm counter for chapter-level
\renewcommand{\thenalg}{\arabic{nalg}} %defines appearance of the algorithm counter
\DeclareCaptionLabelFormat{algocaption}{Algorithme \thenalg} % defines a new caption label as Algorithm x.y

%defines the algorithm listing environment
\lstnewenvironment{algorithm}[1][] {
%increments algorithm number
\refstepcounter{nalg}
%defines the caption setup for: it ises label format as the declared caption label above and makes label and caption text to be separated by a ':'
\captionsetup{labelformat=algocaption,labelsep=colon}
\lstset{ %this is the stype
mathescape=true,
frame=tB,
numbers=left,
numberstyle=\tiny,
basicstyle=\scriptsize,
keywordstyle=\color{black}\bfseries\em,
%add the keywords you want, or load a language as Rubens explains in his comment above.
keywords={, Fonction, Variables, Debut, Fin, Si, Sinon, FinSi, Pour, FinPour, Alors, Faire, retourner, }
numbers=left,
xleftmargin=.04\textwidth,
#1 % this is to add specific settings to an usage of this environment (for instance, the caption and referable label)
}
}
{}

% Document
\begin{document}
    \maketitle
    \newpage
    \tableofcontents
    \newpage

    \section{Introduction}\label{sec:introduction}

    Ce projet a été réalisé par Paul-Emile Watelot et Mathieu Tabary en première année de Master Informatique de Nancy.

    L'objectif de ce projet est de concevoir et mettre en oeuvre un algorithme heuristique de 3-coloriage d'un graphe.

    \section{Complexité du 3-coloriage}\label{sec:complexite-du-3-coloriage}

    \subsection{Donnez un schéma algorithmique déterministe permettant de trouver à coup sûr un 3-coloriage d'un graphe
    $G=(V, E)$ lorsque celui-ci est 3-coloriable}\label{subsec:Q2A}

    On note $0$ la première couleur, $1$ la deuxième couleur et $2$ la troisième couleur.
    \emph{Initialisation} : On fixe la couleur de tous les sommets à 0.
    \emph{Tant qu'il existe une arête reliant deux sommets de la même couleur OU que tous les sommets ont une couleur différente de 2}
    \begin{enumerate}
        \item On ajoute 1 à la couleur du premier sommet.
        \item Si la couleur du premier sommet est supérieure à 2 on la met à 0 et on augmente la valeur du sommet suivant.
        \item Si le sommet suivant à une couleur supérieure à 2 on recommence avec le sommet suivant.
    \end{enumerate}

    Ce schéma algorithmique correspond à tester toutes les combinaisons possibles, même celles dont on est spur qui ne fonctionneront pas.

    \subsection{Évaluez sa complexité en fonction des dimensions du graphe $(n = |V| $ et $ m = |E|)$.}\label{subsec:Q2B}
    S'il existe toujours une arête reliant deux sommets de la même couleur, l'algorithme parcourra $3^n$ fois la boucle tant que.\\
    La vérification de l'existence d'une arête reliant deux sommets de même couleur a comme complexité au pire cas $O(m)$.
    En effet il suffit de parcourir chaque arête et regarder la couleur de ses deux extrémités.\\
    L'incrémentation de la couleur de chaque sommet en moyenne prend $\log_3(m)$ en temps.\\
    Ainsi cet algorithme a une complexité de $O(3^n*m\log_3(m))$, il n'est donc pas applicable en pratique.

    \newpage
    \section{Algorithme heuristique}\label{sec:algorithme-heuristique}

    \subsection{Proposez une stratégie de 3-coloriage d'un graphe
    en expliquant précisément en quoi elle favorise l'obtention d'un coloriage valide}\label{subsec:Q3A}
    On sélectionne le sommet de plus petit degré, on lui attribue une des trois couleurs possibles,
    puis on tente de colorier les sommets voisins avec les deux couleurs restantes (les trois couleurs, sans la couleur attribuée au premier sommet).
    Pour chaque sommet ainsi colorier, on essaie de colorier les sommets voisins avec les couleurs restantes disponibles.

    Cette stratégie favorise l'obtention d'un coloriage valide car on travaille avec un sous-graphe toujours bien colorié.
    En effet, les sommets coloriés sont bien coloriés, et les sommets non coloriés ne sont pas mal coloriés.

    \subsection{Déduisez-en un algorithme polynômial qui cherche un 3-coloriage d’un graphe donné}\label{subsec:Q3B}

    Pour cet algorithme nous allons utiliser des ensembles avec des méthodes pré-définies telles que :
    \begin{enumerate}
        \item \emph{ensemble\_vide()} qui retourne un nouvel ensemble vide. (Opération en temps constant)
        \item \emph{copie(E : ensemble)} qui retourne une copie profonde de l'ensemble E passé en paramètre. (Opération en temps polynomial (taille de l'ensemble copié))
        \item \emph{E + elt} qui ajoute à l'ensemble \emph{E} l'élément \emph{elt} (Opération en temps constant)
        \item \emph{E - elt} qui retire à l'ensemble \emph{E} l'élément \emph{elt} (Opération en temps constant)
    \end{enumerate}

    \newpage
    \begin{algorithm}[caption={3-coloriage}] \label{alg:alg1}
    Fonction 3-coloriage(G : graphe, S : sommet,
        couleur_dispo_depart : ensemble de couleur) : <sommet, ensemble de couleur>
    Variables
        couleur_voisins : ensemble de couleur
        couleur_dispo_fin : ensemble de couleur
        couleur_dispo_successeurs : ensemble de couleur
        res : <sommet, ensemble de couleur>
    Debut
        res $\gets$ <null, null>
        couleur_dispo_fin $\gets$ copie(couleur_dispo_depart)
        couleur_voisins $\gets$ ensemble_vide()
        Si (S est null) Alors
            S = plusPetitSommet(G)
        FinSi
        Pour chaque sommet V dans voisins(G, S) Faire
            Si couleur_voisins ne contient pas V.couleur Alors
                couleur_voisins $\gets$ couleur_voisins + V.couleur
            FinSi
        FinPour

        Pour chaque couleur C dans couleur_dispo_depart Faire
            Si couleur_voisins contient C Alors
                couleur_dispo_fin $\gets$ couleur_dispo_fin - C
            FinSi
        FinPour

        Si couleur_dispo n'est pas vide Alors
            S.couleur $\gets$ couleur_dispo_depart[0]
            couleur_dispo_successeurs $\gets$ ensemble_vide()
            couleur_dispo_successeurs $\gets$ couleur_dispo_successeurs + Rouge
            couleur_dispo_successeurs $\gets$ couleur_dispo_successeurs + Vert
            couleur_dispo_successeurs $\gets$ couleur_dispo_successeurs + Bleu
            couleur_dispo_successeurs $\gets$ couleur_dispo_successeurs - S.couleur
            Pour chaque sommet V dans voisins(G, S) Faire
                res = 3-coloriage(G, V, couleur_dispo_successeur)
            FinPour
        Sinon
            res = <S, couleur_voisins>
        FinSi

        retourner res

    Fin
\end{algorithm}

    \newpage
    \section{Application aux N-cubes}\label{sec:application-aux-n-cubes}

\end{document}
