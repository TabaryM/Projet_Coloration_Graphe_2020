%! Author = Newtton
%! Date = 04/01/2021

% Preamble
\documentclass[a4paper, 11pt]{article}
\usepackage[french]{babel}
\title{Algorithmique et complexité Projet sur les arbres couvrants}
\author{Paul-Emile Watelot \and Mathieu Tabary}
\date{4 janvier 2021}

% Packages
% Symboles
\usepackage{amsmath}

% Graphes
\usepackage{pgf}
\usepackage{tikz}
\usetikzlibrary{positioning}

% Mise en page
\usepackage{multicol}

% Document
\begin{document}
    \maketitle
    \newpage
    \tableofcontents
    \newpage

    \section{Introduction}\label{sec:introduction}

    Ce projet a été réalisé par Paul-Emile Watelot et Mathieu Tabary en première année de Master Informatique de Nancy.

    \section{Complexité du 3-coloriage}\label{sec:complexite-du-3-coloriage}

    \subsection{Donnez un schéma algorithmique déterministe permettant de trouver à coup sûr un 3-coloriage d'un graphe
    $G=(V, E)$ lorsque celui-ci est 3-coloriable}\label{subsec:Q2A}

    On note $0$ la première couleur, $1$ la deuxième couleur et $2$ la troisième couleur.
    \emph{Initialisation} : On fixe la couleur de tous les sommets à 0.
    \emph{Tant qu'il existe une arête reliant deux sommets de la même couleur OU que tous les sommets ont une couleur différente de 2}
    \begin{enumerate}
        \item On ajoute 1 à la couleur du premier sommet.
        \item Si la couleur du premier sommet est supérieure à 2 on la met à 0 et on augmente la valeur du sommet suivant.
        \item Si le sommet suivant à une couleur supérieure à 2 on recommence avec le sommet suivant.
    \end{enumerate}

    Ce schéma algorithmique correspond à tester toutes les combinaisons possibles, même celles dont on est spur qui ne fonctionneront pas.

    \subsection{Évaluez sa complexité en fonction des dimensions du graphe $(n = |V| $ et $ m = |E|)$.}\label{subsec:Q2B}
    S'il existe toujours une arête reliant deux sommets de la même couleur, l'algorithme parcourra $3^n$ fois la boucle tant que.\\
    La vérification de l'existence d'une arête reliant deux sommets de même couleur a comme complexité au pire cas $O(m)$.
    En effet il suffit de parcourir chaque arête et regarder la couleur de ses deux extrémités.\\
    L'incrémentation de la couleur de chaque sommet en moyenne prend $\log_3(m)$ en temps.\\
    Ainsi cet algorithme a une complexité de $O(3^{n*m\log_3(m)})$

    \section{Algorithme heuristique}\label{sec:algorithme-heuristique}

    \section{Application aux N-cubes}\label{sec:application-aux-n-cubes}

\end{document}
