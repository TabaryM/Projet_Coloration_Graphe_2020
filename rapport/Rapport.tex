%! Author = Newtton
%! Date = 04/01/2021

% Preamble
\documentclass[a4paper, 11pt]{article}
\usepackage[french]{babel}
\title{Algorithmique et complexité Projet sur les arbres couvrants}
\author{Paul-Emile Watelot \and Mathieu Tabary}
\date{4 janvier 2021}

% Packages
% Symboles
\usepackage{amsmath}

% Graphes
\usepackage{pgf}
\usepackage{tikz}
\usetikzlibrary{positioning}

% Mise en page
\usepackage{multicol}
\usepackage{hyperref}

\usepackage{color}
\usepackage{listings}
\usepackage{caption}
\usepackage[utf8]{inputenc}
\usepackage[T1]{fontenc}

%Description de pseudocode
%Code trouvé ici : https://tex.stackexchange.com/questions/111116/what-is-the-best-looking-pseudo-code-package/218450#218450
\newcounter{nalg} % defines algorithm counter for chapter-level
\renewcommand{\thenalg}{\arabic{nalg}} %defines appearance of the algorithm counter
\DeclareCaptionLabelFormat{algocaption}{Algorithme \thenalg} % defines a new caption label as Algorithm x.y

%defines the algorithm listing environment
\lstnewenvironment{algorithm}[1][] {
%increments algorithm number
\refstepcounter{nalg}
%defines the caption setup for: it ises label format as the declared caption label above and makes label and caption text to be separated by a ':'
\captionsetup{labelformat=algocaption,labelsep=colon}
\lstset{ %this is the stype
mathescape=true,
frame=tB,
numbers=left,
numberstyle=\tiny,
basicstyle=\scriptsize,
keywordstyle=\color{black}\bfseries\em,
%add the keywords you want, or load a language as Rubens explains in his comment above.
keywords={, Fonction, Variables, Debut, Fin, Si, Sinon, FinSi, Pour, FinPour, Alors, Faire, retourner, }
numbers=left,
xleftmargin=.04\textwidth,
#1 % this is to add specific settings to an usage of this environment (for instance, the caption and referable label)
}
}
{}

% Document
\begin{document}
    \maketitle
    \newpage
    \tableofcontents
    \newpage

    \section{Introduction}\label{sec:introduction}

    Ce projet a été réalisé par Paul-Emile Watelot et Mathieu Tabary en première année de Master Informatique de Nancy.

    L'objectif de ce projet est de concevoir et mettre en oeuvre un algorithme heuristique de 3-coloriage d'un graphe.

    \section{Complexité du 3-coloriage}\label{sec:complexite-du-3-coloriage}

    \subsection{Donnez un schéma algorithmique déterministe permettant de trouver à coup sûr un 3-coloriage d'un graphe
    $G=(V, E)$ lorsque celui-ci est 3-coloriable}\label{subsec:Q2A}

    On note $0$ la première couleur, $1$ la deuxième couleur et $2$ la troisième couleur.\\
    \emph{Initialisation} : On fixe la couleur de tous les sommets à 0.\\
    \emph{Tant qu'il existe une arête reliant deux sommets de la même couleur OU que tous les sommets ont une couleur différente de 2 Faire}
    \begin{enumerate}
        \item On ajoute 1 à la couleur du premier sommet.
        \item Si la couleur du premier sommet est supérieure à 2 on la met à 0 et on augmente la valeur du sommet suivant.
        \item Si le sommet suivant à une couleur supérieure à 2 on recommence avec le sommet suivant.
    \end{enumerate}

    Ce schéma algorithmique correspond à tester toutes les combinaisons possibles, même celles dont on est sûr qui ne fonctionneront pas.

    \subsection{Évaluez sa complexité en fonction des dimensions du graphe $(n = |V| $ et $ m = |E|)$.}\label{subsec:Q2B}
    S'il existe toujours une arête reliant deux sommets de la même couleur, l'algorithme parcourra $3^n$ fois la boucle tant que.\\
    La vérification de l'existence d'une arête reliant deux sommets de même couleur a comme complexité au pire cas $O(m)$.
    En effet il suffit de parcourir chaque arête et regarder la couleur de ses deux extrémités.\\
    L'incrémentation de la couleur de chaque sommet en moyenne prend $\log_3(m)$ en temps.\\
    Ainsi cet algorithme a une complexité de $O(3^n * \left( m + \log_3(m) \right) )$, il n'est donc pas applicable en pratique.

    \newpage

    \section{Algorithme heuristique}\label{sec:algorithme-heuristique}

    \subsection{Proposez une stratégie de 3-coloriage d'un graphe
    en expliquant précisément en quoi elle favorise l'obtention d'un coloriage valide}\label{subsec:Q3A}
    On sélectionne le sommet de plus petit degré, on lui attribue une des trois couleurs possibles,
    puis on tente de colorier les sommets voisins avec les deux couleurs restantes (les trois couleurs, sans la couleur attribuée au premier sommet).
    Pour chaque sommet ainsi colorier, on essaie de colorier les sommets voisins avec les couleurs restantes disponibles.

    Cette stratégie favorise l'obtention d'un coloriage valide car on travaille avec un sous-graphe toujours bien colorié.
    En effet, les sommets coloriés sont bien coloriés, et les sommets non coloriés ne sont pas mal coloriés.

    \subsection{Déduisez-en un algorithme polynômial qui cherche un 3-coloriage d’un graphe donné}\label{subsec:Q3B}

    Pour cet algorithme nous allons utiliser des ensembles avec les méthodes pré-définies suivantes :
    \begin{itemize}
        \item \emph{ensemble\_vide()} qui retourne un nouvel ensemble vide. (Opération en temps constant)
        \item \emph{est\_vide(E : ensemble)} qui retourne un vrai si l'ensemble $E$ est vide, faux sinon. (Opération en temps constant)
        \item \emph{copie(E : ensemble)} qui retourne une copie profonde de l'ensemble E passé en paramètre. (Opération en temps polynomial (taille de l'ensemble copié))
        \item \emph{contient(E : ensemble, elt : élément)} qui retourne vrai si $E$ contient l'élément $elt$, faux sinon. (Opération en temps constant)
        \item \emph{ajout(E, elt)} qui ajoute à l'ensemble \emph{E} l'élément \emph{elt}. (Opération en temps constant)
        \item \emph{retire(E, elt)} qui retire à l'ensemble \emph{E} l'élément \emph{elt}. (Opération en temps constant)
        \item \emph{premier(E)} qui retourne le premier élément de l'ensemble E. (Opération en temps constant)
        \item \emph{taille(E)} qui retourne le nombre d'élément dans l'ensemble E. (Opération en temps polynomial (taille de l'ensemble copié))
    \end{itemize}

    \bigskip
    Nous allons aussi utiliser les méthodes sur les graphes suivantes :
    \begin{itemize}
        \item \emph{sommets(G)} qui retourne la liste des sommets de G\@.
        \item \emph{voisins(G, S)} qui retourne la liste des voisins de S dans G\@.
        \item \emph{degre(G, S)} qui retourne le degré du sommet S dans G\@.
    \end{itemize}

    \bigskip
    Cet algorithme utilisera aussi la méthode \emph{plus\_petit\_sommet(G : graphe)},
    définie ci-dessous, qui retourne le sommet de degré le plus petit du graphe G\@.

    \begin{algorithm}[caption={Plus petit sommet}] \label{alg:pps}
    Fonction plus_petit_sommet(G : $graphe$) : $sommet$
    Variables
        res : $sommet$
        plus_petit_degre : $entier$
    Debut
        plus_petit_degre $\gets$ $\infty$
        Pour chaque sommet S dans sommets(G) Faire
            Si (plus_petit_degre > S.degre) Alors
                plus_petit_degre $\gets$ S.degre
                res $\gets$ S
            FinSi
        FinPour
        retourner res
    Fin
\end{algorithm}
    \begin{algorithm}[caption={3-coloriage}] \label{alg:3color}
    Fonction 3-coloriage(G : graphe, S : sommet,
        couleur_dispo_depart : $ensemble\ de\ couleur$) : <$sommet$, $ensemble\ de\ couleur$>
    Variables
        couleur_voisins : $ensemble\ de\ couleur$
        couleur_dispo_fin : $ensemble\ de\ couleur$
        couleur_dispo_successeurs : $ensemble\ de\ couleur$
        res : <$sommet$, $ensemble\ de\ couleur$>
    Debut
        res $\gets$ <null, null>
        couleur_dispo_fin $\gets$ copie(couleur_dispo_depart)
        couleur_voisins $\gets$ ensemble_vide()
        Si (S = null) Alors
            S $\gets$ plusPetitSommet(G)
        FinSi
        Si (S.couleur $\ne$ null) Alors
            Pour chaque sommet V dans voisins(G, S) Faire
                Si (non contient(couleur_voisins, V.couleur)) Alors
                    ajout(couleur_voisins, V.couleur)
                FinSi
            FinPour
            Pour chaque couleur C dans couleur_dispo_depart Faire
                Si (contient(couleur_voisins, C)) Alors
                    retire(couleur_dispo_fin, C)
                FinSi
            FinPour
            Si (non est_vide(couleur_dispo)) Alors
                S.couleur $\gets$ couleur_dispo_depart[0]
                couleur_dispo_successeurs $\gets$ ensemble_vide()
                ajout(couleur_dispo_successeurs, Rouge)
                ajout(couleur_dispo_successeurs, Vert)
                ajout(couleur_dispo_successeurs, Bleu)
                retire(couleur_dispo_successeurs, S.couleur)
                Pour chaque sommet V dans voisins(G, S) Faire
                    res $\gets$ 3-coloriage(G, V, couleur_dispo_successeur)
                    Si (res $\ne$ null) Alors
                        retourner res
                FinPour
            Sinon
                res $\gets$ <S, couleur_voisins>
            FinSi
        FinSi
        retourner res
    Fin
\end{algorithm}

    \subsection{Proposez un algorithme qui vérifie si un sommet en conflit lors d’une tentative de 3-coloriage
    appartient à une 4-clique.}\label{subsec:Q3C}
    Pour chercher efficacement toutes les 4-cliques on fait une recherche récursive pour avoir un backtracking.
    Voici la fonction récursive utilisée :
    \begin{algorithm}[caption={4-clique}] \label{alg:4clique}
    Fonction 4_clique (G : $graphe$, S1, S2, S3 : $sommet$) : $ensemble\ de\ sommet$
    Variables
        res : $ensemble\ de\ sommet$
    Debut
        Si (S1 = null) Alors
            Pour chaque sommet S dans sommets(G) Faire
                Si (taille(res) < 4) Alors
                    res $\gets$ 4_clique(G, S, null, null)
                FinSi
            FinPour
        Sinon Si (S2 = null) Alors
            Pour chaque sommet V dans voisins(G, S1) Faire
                Si (taille(res) < 4) Alors
                    res $\gets$ 4_clique(G, S, V, null)
                FinSi
            FinPour
        Sinon Si (S3 = null) Alors
            Pour chaque sommet V dans voisins(G, S1) Faire
                Si (taille(res) < 4) Alors
                    // Si V est voisin de S2
                    Si (contient(voisins(G, S2), V)) Alors
                        res $\gets$ 4_clique(G, S1, S2, V)
                    FinSi
                FinSi
            FinPour
        Sinon
            Pour chaque sommet V dans voisins(G, S1) Faire
                Si (taille(res) < 4) Alors
                    // Si V est voisin de S2 et S3
                    Si (contient(voisins(G, S2), V) et contient(voisins(G, S3), V)) Alors
                        res = <S1, S2, S3, V>
                    FinSi
                FinSi
            FinPour
        FinSi
        retourner res
    Fin
\end{algorithm}

    \subsection{Proposez un algorithme qui vérifie qu'un 3-coloriage d'un graphe est valide.}\label{subsec:Q3D}
    \begin{algorithm}[caption={Vérification 3-coloriage}] \label{alg:3color_verif}
    Fonction verification_3_coloriage(G : $graphe$) : $booléen$
        Variables
            res : $booléen$
        Debut
            res = vrai
            Pour chaque sommet S dans sommets(G) Faire
                Pour chaque sommet S dans voisins(G, S) Faire
                    res = res et (S.couleur $\ne$ V.couleur)
                FinPour
            FinPour
            retourner res
        Fin
\end{algorithm}

    \subsection{Évaluez théoriquement la complexité de vos trois algorithmes.}\label{subsec:Q3E}
    \subsubsection{Complexité théorique de l'algorithme de 3-coloriage}\label{subsubsec:Q3E1}
    La recherche du sommet au plus petit degré ne se fait qu'une fois et se fait par un parcours de tous les sommets du graphe.
    Ce qui a comme complexité $O(n)$.

    On effectue une récursion pour tous les sommets, ce qui ajoute un facteur de complexité de $O(n)$.

    Le nombre moyen de voisins par sommet est de $\frac{m}{n}$.
    Donc l'itération sur les voisins d'un sommet a une complexité de $O(\frac{m}{n})$.

    Le nombre maximum de couleur dans la liste des couleurs des voisins d'un sommet est 3 pour un 3-coloriage.
    Donc le parcours de la liste des couleurs des voisins du sommet que l'on colorie a une complexité de $O(3)$.

    En composant toutes ces complexités on obtient la complexité de l'algorithme : $O\left(n+\left(n*\left(\frac{m}{n}+3\right)\right)\right)$.\\
    Cette complexité se simplifie en $O(n+m+3n)$ puis en $O(4n+m)$ qui est une complexité polynomiale.

    \subsubsection{Complexité théorique de l'algorithme de recherche d'une 4-clique dans un graphe}\label{subsubsec:Q3E2}
    La récursion sur tous les sommets du graphe ajoute un facteur de complexité de $O(n)$.

    La récursion sur tous les voisins d'un sommet ajoute un facteur de complexité de $O\left(\frac{m}{n}\right)$.
    Cette récursion est faite trois fois :
    la première pour tous les voisins de S1, la deuxième pour les voisins de S1 et S2 et la troisième pour les voisins de S1, S2 et S3.
    Ces trois récursions se composent avec comme complexité $O($($\frac{m}{n}$)$^3)$

    En composant ces complexités on obtient la complexité de l'algorithme : $O(n*$($\frac{m}{n}$)$^3)$\\
    Cet algorithme est donc polynomial (mais exponentiel en la taille de la clique recherchée).

    \subsubssection{Complexité théorique de l'algorithme de vérification d'un 3-coloriage d'un graphe}\label{subsubsec:Q3E3}
    Cet algorithme effectue un parcours de chaque arête pour tous les sommets, donc chaque arête est parcourue deux fois
    Ce qui a comme complexité $O(2mn)$.\\
    Cet algorithme est donc polynomial.

    \section{Application aux N-cubes}\label{sec:application-aux-n-cubes}

    \subsection{Quelle conjecture peut-on déduire de l'application de notre schéma algorithmique sur les N-cubes}\label{subsec:Q4C}
    En appliquant notre schéma algorithmique au N-cubes allant de 0 à 8, on peut observer qu'ils sont 2-coloriables (et donc 3-coloriables aussi).

    \subsection{Montrer formellement si cette conjecture est vraie ou fausse}\label{subsec:Q4D}
    Si un graphe est biparti, alors il est possible de le colorier avec une couleur par partie, autrement dit, il est 2-coloriable.

    Prouvons donc que les N-cubes sont biparti.

    On peut voir facilement que un 1-cube est biparti, avec dans une partie le sommet 0, et dans l'autre le sommet 1.

    Supposons qu'un k-cube (graphe G) est biparti.
    Construisons un k+1-cube G' à partir de G.
    Les sommets de G' sont tous les sommets de G, ainsi que pour chaque sommet $s_i$ un nouveau sommet $s_{i+2^k}$.\\
    Les arêtes de G' sont toutes les arêtes de G,
    ainsi que pour toutes les arêtes reliant $s_i$ à $s_j$ dans G, on créer une arête reliant $s_{i+2^k}$ à $s_{j+2^k}$.
    Avec $s_i$ dans une des deux partie du graphe G,
    on a $s_i$ dans la même partie dans G' et $s_{i+2^k}$ dans l'autre partie de G'.
    Le k+1-cube ainsi construit est donc aussi biparti.

    Nous venons ainsi de construire récursivement la preuve qu'un N-cube est toujours biparti.

    Étant donné qu'un graphe biparti est 2-coloriable, les N-cubes sont aussi 2-coloriables.

    La conjecture faite précédement est donc vraie.

\end{document}
