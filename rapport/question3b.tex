\begin{algorithm}[caption={3-coloriage}] \label{alg:alg1}
    Fonction 3-coloriage(G : graphe, S : sommet,
        couleur_dispo_depart : ensemble de couleur) : <sommet, ensemble de couleur>
    Variables
        couleur_voisins : ensemble de couleur
        couleur_dispo_fin : ensemble de couleur
        couleur_dispo_successeurs : ensemble de couleur
        res : <sommet, ensemble de couleur>
    Debut
        res $\gets$ <null, null>
        couleur_dispo_fin $\gets$ copie(couleur_dispo_depart)
        couleur_voisins $\gets$ ensemble_vide()
        Si (S est null) Alors
            S = plusPetitSommet(G)
        FinSi
        Pour chaque sommet V dans voisins(G, S) Faire
            Si couleur_voisins ne contient pas V.couleur Alors
                couleur_voisins $\gets$ couleur_voisins + V.couleur
            FinSi
        FinPour

        Pour chaque couleur C dans couleur_dispo_depart Faire
            Si couleur_voisins contient C Alors
                couleur_dispo_fin $\gets$ couleur_dispo_fin - C
            FinSi
        FinPour

        Si couleur_dispo n'est pas vide Alors
            S.couleur $\gets$ couleur_dispo_depart[0]
            couleur_dispo_successeurs $\gets$ ensemble_vide()
            couleur_dispo_successeurs $\gets$ couleur_dispo_successeurs + Rouge
            couleur_dispo_successeurs $\gets$ couleur_dispo_successeurs + Vert
            couleur_dispo_successeurs $\gets$ couleur_dispo_successeurs + Bleu
            couleur_dispo_successeurs $\gets$ couleur_dispo_successeurs - S.couleur
            Pour chaque sommet V dans voisins(G, S) Faire
                res = 3-coloriage(G, V, couleur_dispo_successeur)
            FinPour
        Sinon
            res = <S, couleur_voisins>
        FinSi

        retourner res

    Fin
\end{algorithm}